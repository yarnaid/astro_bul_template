% -*- root: dissertation.tex -*-
%%% Макет страницы %%%
\geometry{a4paper,top=2cm,bottom=2cm,left=2.5cm,right=1cm}
\allsectionsfont{\centering}
\titlespacing*{\chapter}{}{3\baselineskip}{3\baselineskip}

%%% Кодировки и шрифты %%%
\renewcommand{\rmdefault}{ftm} % Включаем Times New Roman
% \renewcommand{\baselinestretch}{1.5}
\usepackage[onehalfspacing]{setspace}% 1,5 интервал
\usepackage[12pt]{extsizes}
\setlength{\headheight}{15pt}


 \acsetup{
 list-type = table,
 list-style = longtable,
 list-heading = subsection*,
 extra-style = comma,
 page-ref = comma
}

%%%%%%%%%%%%%%%%%%%%%
%%% invert colors %%%
%%%%%%%%%%%%%%%%%%%%%
%\pagecolor[rgb]{0.5,0.5,0.5}
%\color[rgb]{1,1,1}

\sisetup
  {
    math-rm=\mathtt,
    text-rm=\ttfamily
  }


%%% Номера страниц по центру вверху
\usepackage{fancyhdr}
\pagestyle{fancy}
\lhead{\normalsize\chaptername\xspace\thesection}
\chead{\normalsize\thepage}
\rhead{}
\lfoot{}
\cfoot{}
\rfoot{}
\renewcommand{\headrulewidth}{0.4pt}
\renewcommand{\footrulewidth}{0pt}

%%% Выравнивание и переносы %%%
\sloppy					% Избавляемся от переполнений
\clubpenalty=10000		% Запрещаем разрыв страницы после первой строки абзаца
\widowpenalty=10000		% Запрещаем разрыв страницы после последней строки абзаца

%%% Библиография %%%
\makeatletter
% \bibliographystyle{utf8gost705u}	% Оформляем библиографию в соответствии с ГОСТ 7.0.5
\renewcommand{\@biblabel}[1]{#1.}	% Заменяем библиографию с квадратных скобок на точку:
\makeatother

%%% Изображения %%%
\graphicspath{ {images/} } % Пути к изображениям


%%% Цвета гиперссылок %%%
\definecolor{linkcolor}{rgb}{0.9,0,0}
\definecolor{citecolor}{rgb}{0,0.6,0}
\definecolor{urlcolor}{rgb}{0,0,1}
\hypersetup{
    colorlinks=false
}

%%% Оглавление %%%
\renewcommand{\cftchapdotsep}{\cftdotsep}




%%% Доп. команды %%%
\providecommand{\setfont}[2]{{\fontfamily{#1}\selectfont #2}}

\providecommand\udk{524.827\xspace}
\providecommand{\alm}{a_{\ell m}}
% \providecommand{\mean}[1]{\left\langle #1\right\rangle}
\providecommand{\Tq}{T(\vec{q})}
\providecommand{\pvec}[1]{\vec{#1}\mum{}ern2mu\vphantom{#1}}


\providecommand{\cmbfast}{\texttt{CMBFAST}\xspace}
\providecommand{\planck}{Planck\xspace}
\providecommand{\Q}{\textsf{Q}\xspace}
\providecommand{\V}{\textsf{V}\xspace}
\providecommand{\W}{\textsf{W}\xspace}
\providecommand{\K}{\textsf{K}\xspace}
\providecommand{\Red}{\color{red}}
\providecommand{\todo}[1]{\textit{\huge \Red #1}}
\providecommand{\noref}{\todo{ПОИСКАТЬ ССЫЛКУ}}
\providecommand{\clmap}{\texttt{cl2map}\xspace}
\providecommand{\skyview}{\texttt{SkyView}\xspace}
\providecommand{\healpix}{\texttt{HEALPix}\xspace}
\providecommand{\python}{\texttt{python}\xspace}
\providecommand{\django}{\texttt{Django}\xspace}
\providecommand{\degr}{$\degree$}
\providecommand{\fdg}{\mathrlap{.}{\degree}}
\providecommand{\fm}{\mathrlap{.}{^m}}
\providecommand{\Kz}{$K(\mathbf{n}, \omega, z, \ell)$\xspace}
\providecommand{\z}{$z$\xspace}
\providecommand{\Alm}{\texttt{Alm}\xspace}
\providecommand{\Cl}{\texttt{Cl}\xspace}
\providecommand{\pointS}{\texttt{PointSource}\xspace}
\providecommand{\pixelmap}{\texttt{PixelMap}\xspace}
\providecommand{\ghz}[1]{\SI{#1}{\GHz}\xspace}
\providecommand{\jy}[1]{\SI{#1}{\jansky}}
\providecommand{\vs}{\texttt{vs}\xspace}
\providecommand{\amin}[1]{\SI{#1}{\arcmin}}
\providecommand{\mum}[1]{\SI{#1}{\micro m}}
\providecommand{\mpc}[1]{\SI{#1}{\mega\pc}}

\DeclareSIUnit\jansky{Jy}
\DeclareSIUnit\pc{pc}



\newtheorem{mydef}{Определение}
\DeclarePairedDelimiter\floor{\lfloor}{\rfloor}
\DeclarePairedDelimiter\brackets{\lbrack}{\rbrack}
\DeclarePairedDelimiter\mean{\langle}{\rangle}
\DeclareMathOperator{\D}{d}




%%%%%%%%%%%%%%%%%%%%%%%%
%%% SHADOWS & STYLES %%%
%%%%%%%%%%%%%%%%%%%%%%%%

\definecolor{spart_bot}{rgb}{.9,.8,.0}
\definecolor{spart_top}{rgb}{.99,.99,.4}

\definecolor{part_bot}{rgb}{0.3,0.7,0.3}
\definecolor{part_top}{rgb}{0.5,1.5,0.5}

\tikzset{
    subpart/.style={
        ellipse,
        top color=spart_top,
        bottom color=spart_bot,
        font={
            \itshape
            \ttfamily
        },
    },
    shadow_st/.style={
        fill=black!50,
        rounded corners=1mm,
    },
    part/.style={
        rectangle,
        rounded corners=3mm,
        top color=part_top,
        bottom color=part_bot,
        font={
            \large
            \ttfamily
            },
    },
    pic_shadow/.style={
        shadow blur extra rounding=3.3pt,
        shadow xshift=.2em,
        shadow yshift=-.2em,
        shadow blur steps=10,
    },
}

\tikzfading[name=fade out, inner color=transparent!0,
  outer color=transparent!100]

\providecommand{\mypic}[2]% file name, args
{   \begin{tikzpicture}
        \node[blur shadow=pic_shadow] {\includegraphics[#1]{#2}};
    \end{tikzpicture}
}
%%%%%%%%%%%%%%%%%%%%%%%%
